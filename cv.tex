%%%%%%%%%%%%%%%%%%%%%%%%%%%%%%%%%%%%%%%%%
% Medium Length Professional CV
% LaTeX Template
% Version 2.0 (8/5/13)
%
% This template has been downloaded from:
% http://www.LaTeXTemplates.com
%
% Original author:
% Trey Hunner (http://www.treyhunner.com/)
%
% Important note:
% This template requires the resume.cls file to be in the same directory as the
% .tex file. The resume.cls file provides the resume style used for structuring the
% document.
%
%%%%%%%%%%%%%%%%%%%%%%%%%%%%%%%%%%%%%%%%%

%----------------------------------------------------------------------------------------
%	PACKAGES AND OTHER DOCUMENT CONFIGURATIONS
%----------------------------------------------------------------------------------------

\documentclass{resume} % Use the custom resume.cls style
\usepackage[UTF8]{ctex}
\usepackage[left=0.75in,top=0.6in,right=0.75in,bottom=0.6in]{geometry} % Document margins
\newcommand{\tab}[1]{\hspace{.2667\textwidth}\rlap{#1}}
\newcommand{\itab}[1]{\hspace{0em}\rlap{#1}}
\name{支龙} % Your name
% \address{黑龙江省哈尔滨市南岗区} % Your address
%\address{123 Pleasant Lane \\ City, State 12345} % Your secondary addess (optional)
\address{(+86)~155~0465~3947 \\ zhilong1550@gmail.com} % Your phone number and email

\begin{document}

%----------------------------------------------------------------------------------------
%	EDUCATION SECTION
%----------------------------------------------------------------------------------------

\begin{rSection}{教育经历}

{\bf 哈尔滨工业大学} \hfill {\em 2015年9月 - 至今} 
\\ 专业:计算机科学与技术 \qquad 专业绩点:3.24/4\\
核心课程:算法设计与分析(89)/快速原型系统设计(90)/操作系统(91)/软件设计与开发实践(86)/计算机视觉(88)/高级语言程序设计(100)/机器学习概论(83)\\
  CET-6 

%Minor in Linguistics \smallskip \\
%Member of Eta Kappa Nu \\
%Member of Upsilon Pi Epsilon \\


\end{rSection}
%----------------------------------------------------------------------------------------
%	TECHNICAL STRENGTHS SECTION
%----------------------------------------------------------------------------------------


% \\ \itab{Process Control (ongoing)} \tab{} \tab{Electrodynamics}

%----------------------------------------------------------------------------------------
%	WORK EXPERIENCE SECTION
%----------------------------------------------------------------------------------------
\begin{rSection}{语言和技能}

\begin{tabular}{ @{} >{\bfseries}l @{\hspace{6ex}} l }
  C/C++, MySQL, Python, HTML, Git 
\end{tabular}

\end{rSection}
\begin{rSection}{项目经历}
\begin{rSubsection}{GAN去模糊(Keras, Python)}{May. 2018}{}

\begin{itemize}

    \item 利用生成式对抗网络(GAN)对图片去模糊,数据集:GOPRO dataset。
    \item 模型的设计中,生成器采用了9层ResNet blocks提高效率,接受模糊图形并进行上采样,对输出进行归一化处理。判别器为一层卷积层,为了改善收敛情况,使用了Wasserstein loss来对整个模型训练。

    \item 生成结果用YOLO进行对象检测,去模糊后能够识别之前无法检测的图像,原物体的的检测程度有了10\%以上的提升。
\end{itemize}

\end{rSubsection}
\begin{rSubsection}{接送通知系统(RFID, C)}{Jul. 2018}{}

\begin{itemize}
    \item 学生管理系统的子系统。该系统基于RFID技术,通过RFID感应器识别家长车载的RFID卡片进行通讯,实现家长接送孩子场景下的数据记录、信息通知等功能。
    \item 主要负责底层逻辑设计,项目代码编写:解决了多个家长接送先后次序问题;缩减等待时间提高效率;设计卡片内编码格式(如高8位代表家长ID,低8位代表车牌号等)。
    \item 系统最终效果:家长在远处刷卡终端即可通知孩子,且能正确处理多次刷卡,误接孩子等问题。

\end{itemize}

\end{rSubsection}

%------------------------------------------------
\begin{rSubsection}{
预测汽车测试时间(Python,XGBoost)}{Jul. 2018}{}

\begin{itemize}
    \item kaggle赛题,要求通过汽车数据特征来预测检测车辆所需时间开销,该数据集只提供匿名标签。

    \item 由于标签匿名,先作假设验证并筛选出对结果影响较大的重要标签。然后对数据进一步处理,如去掉常量标签,利用t检验去掉测试集和训练集均值差较大的标签。为了提高准确性,采用了不同的估计参数进行估计。训练集较小,因此采取交叉检验进行训练。
    \item 模型效果达到前5\%。
\end{itemize}

\end{rSubsection}




\end{rSection}


%	EXAMPLE SECTION
%----------------------------------------------------------------------------------------

%----------------------------------------------------------------------------------------






\end{document}
